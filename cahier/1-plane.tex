\documentclass[cahier-main.tex]{subfiles}
\begin{document}

\chapter{Vectors and Lines the Plane}
\label{ch:one}

\section*{Points and Vectors}

\begin{task}
Write down three distinct points in the plane in proper notation. 
Plot those three points on a single diagram.
\end{task}

\begin{task}
Write down three distinct vectors in the plane in proper notation. 
Your three points from this task should NOT match any of the three points from the previous task.
Plot those three vectors on a single diagram.
\end{task}

\begin{definition}
If $u_1, u_2, \ldots, u_n$ is a collection of vectors, and $\lambda_1, \lambda_2, \ldots$ is a collection of scalars, then the vector formed below is called a \emph{linear combination}\marginnote{Defintion: Linear Combinations} of the $u_i$'s.
\[
\lambda_1 u_1 + \lambda_2 u_2 + \dots \lambda_n u_n
\]
\end{definition}

\begin{task}
Find the sum of your three vectors from the last exercise. Then, choose some order of those three vectors so that they are $u_1$, $u_2$ and $u_3$, and compute the linear combination
\[
3u_1 - 2u_2 + (1/2)u_3.
\] 
\end{task}

\begin{task}
Let's consider the vectors $u=\left(\begin{smallmatrix} 2 \\1 \end{smallmatrix}\right)$, 
$v=\left(\begin{smallmatrix} 1\\ 1 \end{smallmatrix}\right)$, and $w=\left(\begin{smallmatrix} -3\\ 1 \end{smallmatrix}\right)$.
Compute all of the vectors in this list:
\[
\dfrac{u+v}{2}, v-u, v-w, u + \left(\dfrac{v-u}{3}\right), u + \left(\dfrac{3(v-u)}{4}\right)
\]
Then make a single diagram which contains $u$, $v$, $w$ and all of those vectors from the list, plotted as accurately as you can.

What do you notice? Is anything interesting going on?
\end{task}

\begin{task}
Consider the vector $u = \left(\begin{smallmatrix} -4 \\ 2\end{smallmatrix}\right)$. Find a vector $v$ which has the property that $u+v$ is the zero vector, or explain why this is not possible.
\end{task}

\begin{definition}
An equation of the form $\lambda_1 u_1 + \dots \lambda_n u_n = w$, where all of the vectors $u_i$ and $w$ are known, but the scalars $\lambda_i$ are unknown, is called a \emph{linear combination of vectors equation}.
\marginnote{Definition: Linear Combination Equations and their Solutions}
A \emph{solution} to such an equation is 
a collection of scalars which make the equation true.
\end{definition}

\begin{task}
For now, keep $u = \left(\begin{smallmatrix} -4 \\ 2\end{smallmatrix}\right)$. 
Let $v = \left(\begin{smallmatrix} 3 \\ 5 \end{smallmatrix}\right)$.
How many solutions does the linear combination of vectors equation
$\lambda u = v$
have?

How many solutions does the linear combination of vectors equation 
$\lambda u + \mu v = 0$
have? (Here, treat $0$ as the zero vector.)
\end{task}

\begin{task}
We still use the notation $u$ for the vector $u = \left(\begin{smallmatrix} -4 \\ 2\end{smallmatrix}\right)$, but now use $v$ for the vector $v = \left(\begin{smallmatrix} 28 \\ -14\end{smallmatrix}\right)$.
How many solutions does the linear combination of vectors equation $\lambda u = v$
have?

How many solutions does the linear combination of vectors equation 
$\lambda u + \mu v = 0$ have? (Again, treat $0$ as the zero vector.)
\end{task}

\begin{task}
Find the midpoint between the points $P = (4,-2)$ and $Q=(3,5)$. Then find the two points which divide the segment $PQ$ into thirds.

How can vectors make this simpler than it first appears?
\end{task}

\begin{challenge}
Suppose you are given three points in the plane. Let's call them $P$, $Q$, and $R$.
How can you use vectors to (quickly) determine if these three points are collinear?
\end{challenge}

\section*{Parametric Lines in the Plane}

\begin{task}
Write down five different points which lie on the line described parametrically as:
\[
t \mapsto \begin{pmatrix} -5\\3 \end{pmatrix} + t \begin{pmatrix}1\\-1/2\end{pmatrix}.
\]
\end{task}


\begin{challenge}
How many different solutions\marginnote{Recall the definition above.} to the equation
\[
x\begin{pmatrix} 3 \\ 7\end{pmatrix} + y \begin{pmatrix} 6 \\ 14 \end{pmatrix}
= \begin{pmatrix} -3 \\ -7 \end{pmatrix}
\]
can you find? How many are they? Can you find a good way to describe all of the solutions? Is there a natural geometric way to describe all of the solutions? Is there an algebraic way to describe all of the solutions?
\end{challenge}


\begin{task}\label{task:line-through-origin}
Write down a parametric description for the line which passes through the origin $O=(0,0)$ and the 
point $S = (-5,5)$.

Is this the \emph{only} way to write down such a parametric description for that line?
\end{task}


\begin{task}\label{task:line-through-points}
Write down a parametric description for the line which passes through the points below.
\[
 T = (\pi, 0), \qquad J = (0,-\pi)
\]
Now find a different parametric description for that line. (\emph{Hint: Can you find a way to write a parametric description that doesn't use the number $\pi$?})
\end{task}

\begin{task}
Compare the lines from the tasks \ref{task:line-through-origin} and \ref{task:line-through-points}. Do you note anything interesting? How do you know your observation is true?
\end{task}


\begin{task}
For each of the conditions below, either find an example of a $2$-vector $Y$ so that the equation
\[
t\begin{pmatrix}3\\-1\end{pmatrix} = Y
\]
has the given number of solutions, or explain why such an example is not possible.
\begin{compactitem}
\item[a)] exactly zero solutions;
\item[b)] exactly one solution; 
\item[c)] exactly two solutions.
\end{compactitem}
\end{task}



\begin{task}
For each of the conditions below, either find an example of a $2$-vector $Z$ so that the equation
\[
\begin{pmatrix}1\\2\end{pmatrix} + t\begin{pmatrix}-1\\1 \end{pmatrix} = Z
\]
has the given number of solutions, or explain why such an example is not possible.
\begin{compactitem}
\item[a)] exactly zero solutions;
\item[b)] exactly one solution; 
\item[c)] exactly two solutions.
\end{compactitem}
\end{task}

\begin{task}
For each of the conditions below, either find an example of a $2$-vector $Y$ so that the equation
\[
\begin{pmatrix}-2/5\\2\end{pmatrix} + tY = \begin{pmatrix}-1\\1 \end{pmatrix} 
\]
has the given number of solutions, or explain why such an example is not possible.
\begin{compactitem}
\item[a)] exactly zero solutions;
\item[b)] exactly one solution; 
\item[c)] exactly two solutions.
\end{compactitem}
\end{task}

\begin{task}
For each of the conditions below, either find an example of a $2$-vector $Z$ so that the equation
\[
x\begin{pmatrix}-5/3\\1\end{pmatrix} + y\begin{pmatrix}3\\7 \end{pmatrix} = Z
\]
has the given number of solutions, or explain why such an example is not possible.
\begin{compactitem}
\item[a)] exactly zero solutions;
\item[b)] exactly one solution; 
\item[c)] exactly two solutions.
\end{compactitem}
\end{task}

\begin{task}
For each of the conditions below, either find an example of a $2$-vector $Z$ so that the equation
\[
x\begin{pmatrix}-5/3\\1\end{pmatrix} + y\begin{pmatrix}1\\-3/5 \end{pmatrix} = Z
\]
has the given number of solutions, or explain why such an example is not possible.
\begin{compactitem}
\item[a)] exactly zero solutions;
\item[b)] exactly one solution; 
\item[c)] exactly two solutions.
\end{compactitem}
\end{task}

\begin{challenge}
Find an example of four $2$-vectors $X$, $Y$, $Z$, and $W$ so that the equation
\[
aX+bY+cZ = W
\]
has at exactly two solutions, or explain why such an example is not possible.
\end{challenge}

\section*{The Dot Product: Norms and Angles}

\begin{task}
Choose three different $2$-vectors which have neither of their components equal to zero. Call these vectors $u$, $v$, and $w$.
\begin{compactitem}
\item[a)] Compute the norms of $u$, $v$, and $w$.
\item[b)] Compute the dot products $u\cdot v$, $v\cdot w$, and $u\cdot w$.
\item[c)] Find unit vectors $u'$, $v'$, and $w'$ which point in the same directions as $u$, $v$, and $w$, respectively.
\item[d)] Find the angles between each of the pairs, $u$ and $v$, $u$ and $w$, $v$ and $w$ in radians.
\end{compactitem}
\end{task}



\begin{task}
Fix some vector $u$. Draw a picture of $u$ in the plane, and then shade the region of the plane which contains vectors $v$ so that $u\cdot v> 0$.
\end{task}



\begin{task} This task continues our quest for understanding the sign of a dot product geometrically.
\begin{compactitem}
\item[a)] Find an example of two $2$-vectors $v$ and $w$ so that $\left(\begin{smallmatrix}1 \\ 2 \end{smallmatrix}\right)\cdot v =0$ and $\left(\begin{smallmatrix}1 \\ 2 \end{smallmatrix}\right)\cdot w = 0$, or explain why such an example is not possible.

\item[b)] Let $v = \left(\begin{smallmatrix}3\\-1 \end{smallmatrix}\right)$. Find an example of a pair of $2$-vectors $u$ and $w$ such that $v \cdot u < 0$ and $v \cdot w < 0$ and $w \cdot u = 0$, or explain why no such pair of vectors can exist.

\item[c)] Find an example of three $2$-vectors $u$, $v$, and $w$ so that $u \cdot v < 0$ and $u\cdot w < 0$ and $v \cdot w < 0$, or explain why no such example exists.
\end{compactitem}
\end{task}

\begin{task}
What shape is the set of solutions $\left(\begin{smallmatrix} x \\ y \end{smallmatrix}\right)$ to the equation
\[
\begin{pmatrix} 3 \\ 7\end{pmatrix} \cdot \begin{pmatrix} x \\ y \end{pmatrix} = 5?
\] 
That is, if we look at all possible vectors $\left(\begin{smallmatrix} x \\ y \end{smallmatrix}\right)$
which make the equation true, what shape does this make in the plane? Draw this shape.

What happens if we change the vector $\left(\begin{smallmatrix} 3 \\ 7 \end{smallmatrix}\right)$ to some other vector? What happens if we change the number $5$ to some other number?
\end{task}


\begin{task}
\begin{compactitem}
\item[a)] Find an example of a number $c$ so that the equation
\[
\begin{pmatrix} 1 \\ -1 \end{pmatrix} \cdot \begin{pmatrix} x \\ y \end{pmatrix} = c
\]
has the vector $\left(\begin{smallmatrix}4 \\ 7 \end{smallmatrix}\right)$ as a solution, or explain why no such number exists.
\item[b)] Let $v = \left(\begin{smallmatrix}2\\1\end{smallmatrix}\right)$ and $w=\left(\begin{smallmatrix}-3\\4\end{smallmatrix}\right)$. Find an example of a number $c$ so that 
\begin{gather*} 
v \cdot \begin{pmatrix}1\\-1\end{pmatrix} = c \quad\text{ and } \quad w \cdot \begin{pmatrix}1\\-1\end{pmatrix} = c, 
\end{gather*}
or explain why this is not possible.
\item[c)] Let $P = \left(\begin{smallmatrix}-3\\4\end{smallmatrix}\right)$. Find an example of numbers $c$ and $d$ so that 
\begin{gather*} \begin{pmatrix} 2\\-1\end{pmatrix}\cdot P = c \quad\text{ and } \quad \begin{pmatrix} 1\\-1\end{pmatrix}\cdot P = d, 
\end{gather*} 
or explain why no such example is possible.
\end{compactitem}
\end{task}

\section*{Equations of Lines in the Plane}

\begin{task}
Write down an equation for the set of vectors which are all orthogonal to $u=\left(\begin{smallmatrix} 3 \\ -2\end{smallmatrix}\right)$.
\end{task}

\begin{task}\label{task:norm-to-param}
We begin with a line described parametrically by
\[
t \mapsto \begin{pmatrix} 6\\ -\pi \end{pmatrix} + t \begin{pmatrix} 34 \\ -19/3\end{pmatrix}.
\]
\begin{compactitem}
\item[a)] Find a normal vector for this line.
\item[b)] Plot the line and the normal vector you found.
\end{compactitem}
\end{task}

\begin{task}
Find an equation for the line in the Task \ref{task:norm-to-param}.
\end{task}

\clearpage

\begin{task}\label{task:norm-to-eqn}
We begin with a line described by the equation
\[
-3x + y =7.
\]
\begin{compactitem}
\item[a)] Find a normal vector to the line
\item[b)] Plot this line and the normal vector you found.
\end{compactitem}
\end{task}

\begin{task}
Find a parametric description for the line from Task \ref{task:norm-to-eqn}.
\end{task}

\begin{task}
Consider the line described by the equation $4x+7y=3$. Find an equation for the line which is parallel to this one, but passes through the point indicated:
\begin{compactitem}
\item[a)] The origin $O$.
\item[b)] The point $P = (0,-10)$.
\end{compactitem}
\end{task}

\begin{challenge}
Consider the line described by the equation $x -2y = -2$. Find a line which is orthogonal to this one, and passes through the point $Q=(9,-1)$. Can you describe your line with an equation? Can you describe your line parametrically?
\end{challenge}




\end{document}