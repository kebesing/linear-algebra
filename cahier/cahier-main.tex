\documentclass{tufte-book}
\usepackage{subfiles}
\usepackage{amsmath,amssymb,amsthm}

\hypersetup{colorlinks}% uncomment this line if you prefer colored hyperlinks (e.g., for onscreen viewing)

%%
% Book metadata
\title{Elements of Linear Algebra, Volume II\\Cahier des Lignes}
\author[TJH]{Theron J Hitchman}
\publisher{I made it myself.}

%%
% If they're installed, use Bergamo and Chantilly from www.fontsite.com.
% They're clones of Bembo and Gill Sans, respectively.
\IfFileExists{bergamo.sty}{\usepackage[osf]{bergamo}}{}% Bembo
\IfFileExists{chantill.sty}{\usepackage{chantill}}{}% Gill Sans

%\usepackage{microtype}

%%
% For nicely typeset tabular material
\usepackage{booktabs}

%%
% For graphics / images
\usepackage{graphicx}
\setkeys{Gin}{width=\linewidth,totalheight=\textheight,keepaspectratio}
\graphicspath{{graphics/}}

% The fancyvrb package lets us customize the formatting of verbatim
% environments.  We use a slightly smaller font.
\usepackage{fancyvrb}
\fvset{fontsize=\normalsize}

%%
% Prints argument within hanging parentheses (i.e., parentheses that take
% up no horizontal space).  Useful in tabular environments.
\newcommand{\hangp}[1]{\makebox[0pt][r]{(}#1\makebox[0pt][l]{)}}

%%
% Prints an asterisk that takes up no horizontal space.
% Useful in tabular environments.
\newcommand{\hangstar}{\makebox[0pt][l]{*}}

%%
% Prints a trailing space in a smart way.
\usepackage{xspace}

%%
% Prints the month name (e.g., January) and the year (e.g., 2008)
\newcommand{\monthyear}{%
  \ifcase\month\or January\or February\or March\or April\or May\or June\or
  July\or August\or September\or October\or November\or
  December\fi\space\number\year
}


% Prints an epigraph and speaker in sans serif, all-caps type.
\newcommand{\openepigraph}[2]{%
  %\sffamily\fontsize{14}{16}\selectfont
  \begin{fullwidth}
  \sffamily\large
  \begin{doublespace}
  \noindent\allcaps{#1}\\% epigraph
  \noindent\allcaps{#2}% author
  \end{doublespace}
  \end{fullwidth}
}

% Inserts a blank page
\newcommand{\blankpage}{\newpage\hbox{}\thispagestyle{empty}\newpage}

\usepackage{units}

% Typesets the font size, leading, and measure in the form of 10/12x26 pc.
\newcommand{\measure}[3]{#1/#2$\times$\unit[#3]{pc}}


%%% LaTeX Math Stuff
% Theorem-like environments
\theoremstyle{definition}
\newtheorem{task}{Task}
\newtheorem{question}[task]{Question}
\newtheorem{challenge}[task]{Challenge}
\newtheorem*{definition}{Definition}



% Macros for typesetting the documentation
% Generates the index
\usepackage{makeidx}
\makeindex

\begin{document}

% Front matter
\frontmatter

% r.1 blank page
%\blankpage

% v.2 epigraphs
%\newpage\thispagestyle{empty}
%
%\openepigraph{%
%This is where I put an epigraph.
%}{speaker%, {\itshape source}
%}



% r.3 full title page
\maketitle


% v.4 copyright page
\newpage
\begin{fullwidth}
~\vfill
\thispagestyle{empty}
\setlength{\parindent}{0pt}
\setlength{\parskip}{\baselineskip}
Copyright \copyright\ \the\year\ \thanklessauthor

\par\smallcaps{Published by \thanklesspublisher}

\par\smallcaps{sites.uni.edu/theron/la17}

%\par Drop a license statement here.%\index{license}

\par\textit{First printing, \monthyear}
\end{fullwidth}

% r.5 contents
\tableofcontents
%\listoffigures
%\listoftables

%% r.7 dedication
%\cleardoublepage
%~\vfill
%\begin{doublespace}
%\noindent\fontsize{18}{22}\selectfont\itshape
%\nohyphenation
%This is the dedication page. You should know that dedication is an important part of 
%being successful in mathematics. Imagination plays a role, but mostly mathematicians are people who like the work just enough to fail to give up when things become challenging. 
%
%Mathematics challenges everyone at one point or another. Being confused and stuck is 
%a mathematician's natural state. The professionals just got used to it. 
%
%So, stick with dedication, persistence, and perseverance. Those you can build. Don't give up. Talk about math with whoever will join in. All the cool ideas are buried under hard work.
%\end{doublespace}
%\vfill
%\vfill


% r.9 introduction
\cleardoublepage
\chapter*{Introduction}

This is a \emph{workbook}. 
It is a collection of tasks that you should do to try to learn linear algebra for yourself.

It is not a \emph{textbook}. We don't have a textbook, as such things are understood today. We have a primer, the first volume of this set: \emph{Livre des Lignes}, which
should serve as a basic source for reading about the main ideas, and we have this workbook. The books are separate so that you may have them both open at once. I hope this works.


It is important to focus on your own understanding when working on these tasks. At every stage possible, you should ask yourself, ``Self, how do I know this?'' and maybe also, ``So, Self, do I know this for sure? Or do I still have doubts?''

If you are feeling at a loss for how to explain your thinking, try relying on simple geometric models and reasoning. It turns out that most of even the fanciest linear algebra is done that way. So, ``Self, can you draw the relevant picture? Does that help?''

Note that some of these items are labeled as \emph{tasks} and some are labeled as \emph{challenges}. A task is something I think you should be able to do after having read the \emph{Livre des Lignes}: the relevant procedure or problem is considered there somewhere. But a challenge is something that will require some independent thinking: not all of the work will correspond directly to the development in the \emph{Livre}, and you might need to use some new idea, or new combination of ideas. Usually, the challenges are there to help set up ideas we will return to later.


Good luck. I'll see you in class, and I welcome your visits to office hours to talk about mathematics.


%%
% Start the main matter (normal chapters)
\mainmatter

\subfile{1-plane}
\subfile{2-space}
\subfile{3-views}
\subfile{4-systems}
\subfile{5-spaces}


%%
% The back matter contains appendices, bibliographies, indices, glossaries, etc.
%\backmatter
%\bibliography{sample-handout}
%\bibliographystyle{plainnat}
%\printindex

\end{document}

