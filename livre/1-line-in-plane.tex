\documentclass[00-livre-main.tex]{subfiles}
\begin{document}

\chapter{The Equation of a Line in the Plane}


Let's recall the idea of \emph{the plane} from classical geometry: the plane is like a flat sheet of drawing paper, which extends indefinitely in all directions without bound. 
It is the playground for lots of serious considerations from high school: points, lines, circles, triangles, rectangles, and various other doodles live in it. 
I say \emph{in} it rather than \emph{on} it, because all of those objects really have their existence inside the plane. 
If we were to say ``on the plane'' then one might think of them as sitting on top of the paper, where a light breeze might move them about. 
No, those things live inside the plane just as sure as you and I live inside the universe.

And while all of that is evocative and romantic, it doesn't make doing mathematics any easier. 
Our aim in this first chapter is to do some concrete mathematics: we want to figure out how to describe a single line in the plane very carefully. 
To do this, we will use tools that Ren\'{e} Descartes taught us: coordinates. 
Better yet, we will use an update of the idea and introduce \emph{vectors}. 
Our work has to rely on something, so at some points we will make use of geometry facts you learned in high school. 
But for all of the points and vectors, angles and dot products, we will go straight to the heart of a single important question:

\begin{quotation}
\textbf{\large How can we clearly describe a single line in the plane with an equation?}
\end{quotation}

\clearpage

\section*{Points, Vectors, and Vectors}


You have likely seen the idea of \emph{Cartesian coordinates} on the plane before. To be clear, let's set things down carefully. In the plane, we choose a pair of perpendicular lines which meet at a point $O$. This special point is called the \emph{origin}. Then, we choose a point $X$ on one of the lines and draw the circle centered at $O$ which passes through $X$. Note that this circle meets our two lines in two points each, four points total, one of which is the point $X$. Then, from $X$, we rotate around the circle by a quarter turn counterclockwise until we hit one of the points on the other line. This new point we will call $Y$. Are you drawing with me? Here is my picture so far.

\begin{figure}[h!]
\centering
\begin{tikzpicture}[scale=2]
\draw (-1,4/3) -- (1,-4/3);
\draw (-3/2,-9/8) -- (3/2,9/8);
\draw (0,0) circle [radius=3/4];
\draw[->] (-50:1) arc (-50:35:1);
\node [right] at (1.1,0) {rotate counterclockwise};
\node [left] at (0,0) {$O$};
\node [below] at (9/20,-3/5) {$X$};
\node [above] at (3/5,9/20) {$Y$};
\end{tikzpicture}
\end{figure}

We call the line $OX$ the \emph{$x$-axis} and the line containing $Y$ the \emph{$y$-axis}. Here comes the amazing part: we declare the circle we used to be of \emph{unit size}, and make the lines $OX$ and $OY$ into number lines! The important part is that the point $O$ should represent $0$ on both number lines, and the points $X$ and $Y$ should each represent $1$ on their lines. So, instead of marking things with $O$, $X$, and $Y$, we put down a mark where $X$ and $Y$ are and label them with $1$'s, and add little arrows marked with $x$ and $y$ near the positive ``ends'' of the lines $OX$ and $OY$ in our diagram.

\begin{figure}[h!]
\centering
\begin{tikzpicture}[scale=2]
\draw[->] (-1,4/3) -- (1,-4/3);
\draw[->] (-3/2,-9/8) -- (3/2,9/8);
\draw[fill] (9/20,-3/5) circle [radius=0.03];
\draw[fill] (3/5,9/20) circle [radius=0.03];
\node [below] at (9/20,-3/5) {$1$};
\node [above] at (3/5,9/20) {$1$};
\node [right] at (1,-4/3) {\small $x$};
\node [left] at (3/2,9/8) {\small $y$};
\end{tikzpicture}
\end{figure}

Note that above I have done something a bit silly and let the picture just fall on the paper in an unusual way. I really mean unusual as ``not usual.'' The usual way arranges the lines on the paper to match our expected horizontal ($x$) and vertical ($y$) directions. This isn't actually required, but it is what everyone always does. So the picture looks more like this one.


\begin{figure}[h]
\centering
\begin{tikzpicture}[scale=1.25]
\draw[->] (-2,0) -- (2,0) node[below] {\small $x$};
\draw[->] (0,-2) -- (0,2) node[left] {\small $y$};
\draw[fill] (1,0) circle [radius=0.03] node[below] {$1$};
\draw[fill] (0,1) circle [radius=0.03] node[right] {$1$};
\end{tikzpicture}
\end{figure}

\section*{Lines as Parametric Objects}

\begin{compactitem}
\item vector algebra in R2
\item lines in R2 as parametric objects: through the origin, and not
\item sketching lines in R2
\end{compactitem}

\section*{Lengths and Angles in the plane}
\begin{compactitem}
\item lengths, angles, and the dot product in R2
\item normal vector to a line in R2
\item duality
\end{compactitem}


\begin{compactitem}
\item points vs physics vectors vs math vectors
\item the idea of R2
\item sketching points and vectors in the plane
\end{compactitem}


\section*{Lines and Equations}
\begin{compactitem}
\item equation of a line, three methods: elimination, from two pts via similar triangles, from geometry of dot product
\item families of parallel lines
\item sketching a line from an equation
\end{compactitem}

\section*{Conclusion: The Big Theorem}


\end{document}