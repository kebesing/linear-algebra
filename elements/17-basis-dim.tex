\documentclass[elementsmain.tex]{subfiles}
\begin{document}
\section{Bases and Dimension}

Here are some important, and related questions:
\begin{itemize}
\item How big is a subspace, anyway? In what way can we coherently talk about the size of some subspace of $\R^n$? 
\item It seems that describing a subspace as a span (like a column space) is a useful thing. But usually there are multiple ways to pick a spanning set for a given subspace. Is there some sense in which one spanning set might be better than another?
\end{itemize}

We'll now see how to answer these questions, in reverse order. First, to describe a subspace, we will want a set that satisfies a sort of Goldilocks principle. Our set should be big enough to be a spanning set, so that it describes all of the independent directions one can move in the subspace. But also small enough that it is linearly independent, so that it doesn't contain any redundant information.


\begin{definition}[Basis]
Let $\mathcal{S}$ be a subspace of $\R^n$. A set of vectors $\{v_1, \dots, v_k\}$, all of them taken from $\mathcal{S}$, is called a \emph{basis of $\mathcal{S}$} if its both
\begin{compactitem}
\item a linearly independent set, and
\item a spanning set for $\mathcal{S}$.
\end{compactitem}
\end{definition}

By the way, the plural of ``basis'' is ``bases.'' This is a constant trouble for new learners and spellers.

If a basis is the right kind of set to describe a subspace, we can use the size of the basis as a proxy for the size of the subspace. This is the linear algebra specific concept of dimension.

\begin{definition}[Dimension]
Let $\mathcal{S}$ be a subspace of $\R^n$. The \emph{dimension of $\mathcal{S}$}, denoted
$\dim{\mathcal{S}}$, is the number of vectors in a basis of $\mathcal{S}$.
\end{definition}

There is an important theorem to know here, but its proof is really tedious and annoying, so we'll skip that.

\begin{theorem}
Let $\mathcal{S}$ be a subspace of $\R^n$. Then $\mathcal{S}$ has a basis consisting of a finite set of vectors. Moreover, any two bases of $\mathcal{S}$ have the same number of elements. Thus, the dimension of $\mathcal{S}$ makes sense.
\end{theorem}


\clearpage
\subsection*{Exercises}

\begin{exercise} This tasks has two parts, each concerning the subspace of $\R^3$:
\[
\mathcal{S}_0 = \left\{ \begin{pmatrix} x \\ y \\ z \end{pmatrix} \,\middle|\, z=0 \right\}.
\]
\begin{compactitem}
\item[a)] Is the following set of vectors a basis of the subspace $\mathcal{S}_0$?
\[
\left\{ \begin{pmatrix} 3 \\ 4 \\ 0 \end{pmatrix}, \begin{pmatrix} -2 \\ -5 \\ 0\end{pmatrix}, \begin{pmatrix} 5 \\ 5 \\ 0\end{pmatrix} \right\}
\]
Why or why not?
\item[b)] Is the following set of vectors a basis of the subspace $\mathcal{S}_0$?
\[
\left\{ \begin{pmatrix} 4 \\ 2 \\ 0 \end{pmatrix}\right\}
\]
Why or why not?
\end{compactitem}
\end{exercise}

\begin{exercise} This question has two parts. 
\begin{itemize}
\item[a)] Give three different examples of bases of $\R^3$. What is the dimension of $\R^3$?
\item[b)] What is the maximal dimension of a subspace of $\R^4$? How do you know?
\end{itemize}
\end{exercise}

\begin{exercise}
Consider the subspace of $\R^5$ defined as the solution set of the following homogeneous system of linear equations.
\[
\left\{\begin{array}{rrrrrrrrrrr}
 x_1 & + &  2x_2 & - &  3x_3 & + & 2x_4 & - &  4x_5 & = & 0 \\ 
2x_1 & + &  4x_2 & - &  5x_3 & + &  x_4 & - &  6x_5 & = & 0 \\ 
5x_1 & + & 10x_2 & - & 13x_3 & + & 4x_4 & - & 16x_5 & = & 0
\end{array}\right. 
\]
Find a basis for this subspace, and then determine its dimension.
\end{exercise}

\begin{exercise}
Consider the subspace of $\R^3$ defined as the span of the following set of vectors. 
\[
\left\{ \begin{pmatrix} -1 \\ 2 \\ 1\end{pmatrix}, \begin{pmatrix} 7\\3 \\4 \end{pmatrix}, \begin{pmatrix} 2\\ -4 \\ 2 \end{pmatrix}, \begin{pmatrix} 6 \\ 5 \\ 5 \end{pmatrix} \right\}
\]
Find a basis for this subspace, and then determine its dimension.
\end{exercise}

\begin{exercise} Find a basis for the null space of the following matrix. Then determine the dimension of this null space. (This null space lives in some $\R^n$. Which one?)
\[
A = \begin{pmatrix} 5 & 3 & 2 & 0 \\ 1 & -1 & 3 & 2 \end{pmatrix}
\]
\end{exercise}

\begin{exercise} Find a basis for the span of the following four vectors in $\R^6$. Work in two ways: first, use the casting out algorithm to look for linear independence; then use the row algorithm. 
\[
v_1 = \begin{pmatrix} 101 \\   -35 \\    2 \\   17 \\   9 \\ -4 \end{pmatrix}, 
v_2 = \begin{pmatrix}   1 \\     3 \\   88 \\   47 \\   6 \\ -2 \end{pmatrix}, 
v_3 = \begin{pmatrix}  296 \\ -126 \\ -610 \\ -278 \\ -15 \\  2 \end{pmatrix}, 
v_4 = \begin{pmatrix} 4 \\ 9 \\ 17 \\ 2 \\ -35 \\ 101 \end{pmatrix}, 
\]
How do the two results compare? Why might you want to do one process or the other?
\end{exercise}

\begin{exercise} In general, is there a way to describe the dimension of the column space of a matrix? Make several examples and figure out what is going on.
\end{exercise}

\begin{exercise} In general, is there a way to describe the dimension of the null space of a matrix? Make several examples and figure out what is going on.
\end{exercise}

\clearpage
\end{document}